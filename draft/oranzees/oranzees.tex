\documentclass[9pt,twocolumn,twoside,]{pnas-new}

%% Some pieces required from the pandoc template
\providecommand{\tightlist}{%
  \setlength{\itemsep}{0pt}\setlength{\parskip}{0pt}}

% Use the lineno option to display guide line numbers if required.
% Note that the use of elements such as single-column equations
% may affect the guide line number alignment.


\usepackage[T1]{fontenc}
\usepackage[utf8]{inputenc}


\templatetype{pnasresearcharticle}  % Choose template

\title{Copying is not required for ape cultures}

\author[a,1]{Alberto Acerbi}
\author[b]{William Snyder}
\author[b]{Claudio Tennie}

  \affil[a]{Centre for Culture and Evolution, Division of Psychology, Brunel
University London, Uxbridge, UB8 3PH, United Kingdom}
  \affil[b]{Faculty of Science, Department for early prehistory and quaternary
ecology, University of Tübingen, Schloß Hohentuebingen, Burgsteige 11,
72070, Tübingen, Germany}


% Please give the surname of the lead author for the running footer
\leadauthor{Acerbi}

% Please add here a significance statement to explain the relevance of your work
\significancestatement{Human culture is cumulative: it grows in complexity and efficiency,
drawing on the innovations of previous generations. In contrast,
traditions of other primates, while also based on social information,
are cumulative to a minimum degree, if any. It has been proposed that
human cumulative culture depends on our ability to accurately transmit
and preserve information. At the same time, researchers have claimed
that also non-human primates can copy others with high-fidelity. This
raises a puzzling question: why are their cultures not cumulative? We
show, through computer simulations, that the behavioural patterns used
to prove the existence of copying-based culture in wild ape populations
can be reproduced, in realistic conditions, without copying, suggesting
a reason why ape cultures are not cumulative.}


\authorcontributions{AA designed the research, developed the model and analyse the data, and
wrote the paper. WS contributed the artworks for Figure 2. CT designed
the research and wrote the paper.}



\correspondingauthor{\textsuperscript{} }

% Keywords are not mandatory, but authors are strongly encouraged to provide them. If provided, please include two to five keywords, separated by the pipe symbol, e.g:
 \keywords{  cultural transmission |  cultural evolution |  cumulative culture |  non-human great ape culture |  individual-based models |  animal culture  } 

\begin{abstract}
We developed an individual-based model inspired by (1) to test whether
population-level distributions of behavioural traits suggestive of
cultures could emerge without the existence of high-fidelity copying. We
reproduce several details, including realistic demographic and spatial
features, and parametrised the range of genetic and ecological
variations between the populations. We also parametrised the degree to
which individual innovations are socially-mediated, i.e.~the degree to
which the probability for an individual to innovate a behaviour is
dependent on the frequency of the same behaviour in the population. Our
results show that, under realistic values of the main parameters, namely
null to medium importance of genetic variation, medium to high
importance of ecological variation, and various values of social
influence on innovations, we can reproduce distributions of behavioural
traits used to prove the existence of ape cultures. Our model reproduces
other details of the behavioural patterns found in wild population, such
as the proportion of behaviours present in all sites, not habitual in
all site, or absent because of ecological factors, as well as a
correlation between population size and the number of cultural traits in
each site. Overall, our results suggest that ape cultures can emerge
without copying, provide an explanation to why they are not cumulative.
\end{abstract}

\dates{This manuscript was compiled on \today}
\doi{\url{www.pnas.org/cgi/doi/10.1073/pnas.XXXXXXXXXX}}

\begin{document}

% Optional adjustment to line up main text (after abstract) of first page with line numbers, when using both lineno and twocolumn options.
% You should only change this length when you've finalised the article contents.
\verticaladjustment{-2pt}

\maketitle
\thispagestyle{firststyle}
\ifthenelse{\boolean{shortarticle}}{\ifthenelse{\boolean{singlecolumn}}{\abscontentformatted}{\abscontent}}{}

% If your first paragraph (i.e. with the \dropcap) contains a list environment (quote, quotation, theorem, definition, enumerate, itemize...), the line after the list may have some extra indentation. If this is the case, add \parshape=0 to the end of the list environment.

\acknow{This project has received funding from the European Research Council
(ERC) under the European Union's Horizon 2020 research and innovation
programme (grant agreement n° 714658; STONECULT project). We would like
to than Mima Batalovic for the support prvoided, and Elisa Bandini, Alba
Motes Rodrigo, and Jonathan Reeves for comments on earlier versions of
the manuscript.}

Cumulative culture, the transmission and improvement of knowledge,
technologies, and beliefs from individual to individual, and from
generation to generation, is key to explain the extraordinary ecological
success of our species (2, 3). Which cognitive abilities underpin
human's cumulative cultural capacities, and how these abilities affect
the evolution of culture itself are among the most pressing questions of
evolutionary human science.

Many species are able to at least use social cues to modify their
behaviour. Various primates have been shown to posses complex traditions
that are socially influenced (1, 4, 5). Humans, in contrast, have
cumulative culture. While there are various definitions of cumulative
culture (6), some of its characteristics are broadly accepted.
Cumulative culture requires the accumulation of cultural traits (more
cultural traits are present at generation \emph{g} than at time
\emph{g-1}), their improvement (cultural traits at generation \emph{g}
are more effective than at generation \emph{g-1}), and ratcheting (the
innovation of cultural traits at generation \emph{g} depends on the
presence of other traits at generation \emph{g-1}) (7).

While not all human culture needs to be supported by faithful copying
(8), our \emph{cumulative} culture depends on an ability to accurately
transmit and preserve new information. Experiments have indeed shown
that humans are capable of copying, and that they routinely do it in all
known societies. More controversial is the claim that other species
copy. Arguments regarding the existence of non-human great ape cultures
based on copying raise a puzzling question: if other ape species can and
do copy, why do they not develop cumulative cultures? There are only two
possible answers to this question: either they do not copy, or copying
does not automatically lead to cumulative culture.

Primatologists have claimed the existence of ape cultures based on the
ability of copying faithfully drawing on observations conducted on wild
populations. For example, researchers examined the population-level
distribution of behavioral traits in populations of chimpanzees across
seven sites, and argued that the inter-site differences in the frequency
of behaviors proved the existence of copying-based cultures in these
populations (1). We developed an individual-based model to assess
whether these patterns actually justify the conclusion that copying is
the underlying learning mechanism. We reproduced several details of the
original study, including realistic demographic and spatial features,
and effects of ecological availability and genetic predisposition, to
investigate whether an equivalent distribution of behavioral traits
could emerge in absence of any copying. While our simulated species,
\emph{oranzees}, can be influenced by social cues (widespread in the
animal kingdom, and certainly present in all apes), we explicitly did
not model any copying.

Our results show that, under realistic values of the main parameters, we
can reproduce the distribution of behavioral traits found in (1),
without any copying required. In other words, as oranzees can and do
show cultural patterns resembling wild ape patterns, this suggests that
such patterns do not constitute evidence that copying must have taken
place.

\section*{Materials and methods}\label{materials-and-methods}
\addcontentsline{toc}{section}{Materials and methods}

We built an individual-based model that reproduces a world inhabited by
six populations of ``oranzees'', a hypothetical ape species. The model
is spatially explicit: the oranzees populations are located at relative
positions analogous to the six chimpanzees sites in (1). This is
important to determine the potential genetic predispositions and
ecological availabilities associated with their possible behaviours (see
below). Population sizes are also taken from the sites in (1). Following
(9), we use data from (10), and we define population sizes as
\(N=\{20;42;49;76;50;95\}\).

Oranzees are subject to an age-dependent birth/death process, broadly
inspired by descriptions in (11). A time step \(t\) of the simulation
represents a month in oranzees' life. From when they are 25 years old
(\(t=300\)), there is a 1\% probability an oranzee will die each month,
or they die when they are 60 years old (\(t=720\)). The number of
individuals in the population is fixed, so each time an oranzee dies it
is replaced by a newborn.

A newborn oranzee does not yet show any behaviour. Behaviours can be
innovated at each time step. The process of innovation is influenced by:
(i) the oranzees `state', which depends on the behaviours an individual
already possesses, (ii) the frequency of the behaviours already present
in the population (``socially-mediated reinnovation'' in (12)), and
(iii) the genetic propensity and ecological availability locally
associated to the behaviour. At the beginning of the simulations, the
populations are randomly initialised with individuals between 0 and 25
years old.

\subsection*{Oranzee's behaviours and state}\label{format}
\addcontentsline{toc}{subsection}{Oranzee's behaviours and state}

In the oranzees world, 64 behaviours are possible. Behaviours are
divided into two categories: 32 social and 32 food-related behaviours.
These figures where chosen to resemble the behavioural categories
considered in (1). Behaviours serve oranzees to fulfill various goals.
Oranzees have a `state' that is based on how many goals are fulfilled in
the two main categories of social and food-related behaviors.

In the case of social behaviours, we assume four sub-categories (`play',
`display', `groom', `etc.'courtship', the names are only suggestive),
each with eight possible different behaviours that serve the same goal.
A goal is considered fulfilled if an oranzee has at least one behaviour
out of the eight in the sub-category. Oranzees have a `state' that is
based on how many of the four goals are fulfilled. An oranzee has a
state value of \(0.25\) if, for example, it has at least one behaviour
in the category `play', and none of the others, and a state value of
\(1\) if there is at least one behaviour in each sub-category.
\(p_\text{social}\), the probability to innovate a social behaviour, is
drawn from a normal distribution with mean equal to
\(1-state_\text{social}\).

Food-related behaviours are analogously divided into sub-categories.
Differently from social behaviours, there is a variable number of
behaviours in each sub-category. In addition, sub-categories are
associated to two different `nutrients', \emph{Y} and \emph{Z}. Here
individuals need to balance their nutritional intake, so that their
optimal diet consist in a roughly equal number of food for one and the
other nutrient. The state, for food-related behaviours, depends on the
total amount of food ingested \emph{and} on the balance between
nutrients. The state is calculated as the sum of each sub-category
fulfilled (as above, for this to happen there needs to be at least one
behaviour present) minus the difference between the number of
sub-categories providing nutrient \emph{Y} and the number of
sub-categories providing nutrient \emph{Z}. We normalize the state
between \(0\) and \(1\), and, as above \(p_\text{food}\) is then
calculated as \(1-state_\text{food}\).

\subsection*{Socially-mediated reinnovation}\label{format}
\addcontentsline{toc}{subsection}{Socially-mediated reinnovation}

At each time step, all oranzees have a probability of innovation for
social and food-related behaviours calculated as described above. The
specific behaviour an oranzee will innovate depends both on the
frequency of the behaviours already present in the population, and on
the ecological availability and genetic propensity associated to the
behaviour. A further parameter of the model, \(S\), controls the
probability that each reinnovation is socially-mediated (12). When a
reinnovation is socially-mediated, the probability of innovating each
behaviour \(B_i\) is weighted by its proportional instances in the
population, among the behaviours of the same category, so that common
behaviours are more likely to be reinnovated.

When the reinnovation is not socially-mediated, the probability of
innovating each behaviour is random. Only one behaviour per category can
be innovated at each time step.

\subsection*{Genetic propensity and ecological
availability}\label{format}
\addcontentsline{toc}{subsection}{Genetic propensity and ecological
availability}

The behaviour selected in the previous step is then innovated or not
according to its genetic propensity and, in case of food-related
beahviours, ecological availability.

Genetic propensity is a probability \(p_g(0,1)\), assigned independently
for each of the 64 behaviours. A parameter of the model, \(\alpha_g\),
determines the probability that the genetic propensity of each behaviour
is equal for all the six populations or whether is different. If the
probability is equal, \(p_g\) is randomly drawn. If it is different, we
assign the propensity using a geographical gradient. We choose a random
point and calculate its distance to each population. Distances are then
transformed to \(p_g\) by rescaling them between 0 and 1, so that for
the farther site \(p_g=0\) i.e.~the associated behaviour will be
impossible to express (see SI). Notice that \(\alpha_g=0\) does not mean
that there are no genetic influences on the behaviour, but that there
are no \emph{differences} between the populations with regard to this
aspect.

Ecological availability is a probability \(p_e(0,1)\) that represents
the likelihood of finding a resource, or its nutritional value, in each
site. Ecological availability is assigned only to food-related
behaviours, and it is calculated in the same way of \(p_g\), using the
parameter \(\alpha_e\) to determine the probability of ecological
availability being different in the six populations.

\subsection*{Model's output}\label{format}
\addcontentsline{toc}{subsection}{Model's output}

We run simulations for \(t_\text{max}=6000\) (corresponding to 500 years
of oranzee-time). For each simulation, following (1), we classify each
behaviour, in each population, as:

\begin{itemize}
\item
  \emph{customary}: a behaviour observed in over 50\% of individuals in
  at least one age class (see SI for how age classes are defined in our
  model).
\item
  \emph{habitual}: a behaviour observed in at least two individuals
  across the population.
\item
  \emph{present}: a behaviour observed in at least one individual across
  the population.
\item
  \emph{absent}: a behaviour not observed even once in the population.
\item
  \emph{ecological explanations}: a behaviour that is absent because of
  complete lacking of local ecological availability (i.e., in our model,
  associated to \(p_e=0\)).
\end{itemize}

Notice the last category in (1) (\emph{unknown}, i.e. ``the behaviour
has not been recorded, but this may be due to inadequacy of relevant
observational opportunities'') does not apply in our case, because we
have complete knowledge of the output of the simulations.

Finally, to test how well our model compares to the results in wild
apes, we calculate the same ``patterns'' described in (1):

\begin{itemize}
\item
  \emph{A}: behaviour absent at no site.
\item
  \emph{B}: behaviour not achieving habitual frequencies at any site.
\item
  \emph{C}: behaviour for which any absence can be explained by local
  ecological factors.
\item
  \emph{D}: behaviour customary or habitual at some sites yet absent at
  others, with no ecological explanation, i.e.~behaviours defined as
  ``cultural''.
\end{itemize}

Further details of the model implementation and of how outputs are
processed are available in SI. The full code of the model allowing to
reproduce all our results, plus a detailed description of the model
development is available in a dedicated GitHub repository, at
\url{https://github.com/albertoacerbi/oranzees}.

\section*{Results}\label{results}
\addcontentsline{toc}{section}{Results}

We are particularly interested in the realistic parameter conditions of
moderate to high environmental variability (i.e. \(\alpha_e\) from 0.5
to 1) and zero to moderate genetic differences (i.e. \(\alpha_g\) from 0
to 0.5). We ran 20 simulations for each combination (for a total of 600
runs). For all, reinnovation is socially-mediated (\(S=1\)). The results
show that various combinations of parameters produces a number of
cultural behaviours (pattern \emph{D}) consistent with the 38 found in
(1), in absence of any explicit copying mechanism implemented (see
Figure \ref{Figure1}). In Figure \ref{Figure2}, we reproduce the output
of a run where 38 cultural behaviours were found, and how they were
classified in each of the six simulated populations, using a
visualisation inspired by (1).

\begin{figure*}[h!]
\begin{center}
\includegraphics[width=17.8cm]{figures/figure_1.pdf}
\caption{Number of cultural traits in oranzees, when varying ecological and genetic diversity. Red colour indicates simulation runs that produced more than 38 cultural behaviours; blue colour indicates simulation runs that produced less than 38 cultural behaviours. For all simulations, $S=1$, $\alpha_e$ and $\alpha_g$ as indicated in the plot. $N=20$ runs for each parameters combination.}
\label{Figure1}
\end{center}
\end{figure*}

\begin{figure*}[h!]
\begin{center}
\includegraphics[width=13.8cm]{figures/figure_2.pdf}
\caption{Example of a simulation run that produces 38 cultural behaviours ($S=1$, $\alpha_e=0.8$, and $\alpha_g=0.2$). Color icons indicate customary behaviors; circular icons, habitual; monochrome icons, present; clear, absent;  horizontal bar, absent with ecological explanation. The names of the behaviours are only suggestive, see SI for a complete list.}
\label{Figure2}
\end{center}
\end{figure*}

We also analysed the effect of the parameter \(S\) (proportion of
socially-mediated reinnovations), in three conditions (see Figure
\ref{Figure3}): (a) no genetic differences and intermediate ecological
differences (compare to the high-left corner of Figure \ref{Figure1},
where with \(S=1\) simulations produce less than 38 cultural
behaviours), (b) one of the conditions that produce good match with (1),
namely \(\alpha_e=0.8\) and \(\alpha_g=0.2\), and (c) intermediate
genetic differences and high ecological differences (compare to the
low-right corner of Figure \ref{Figure1}, where with \(S=1\) simulations
produce more than 38 cultural behaviours). As expected, decreasing
\emph{S}, decreases the number of cultural behaviours. Conditions where,
with \(S=1\), there were more than 38 cultural behaviours could still
produce results analogous to (1), given that not all reinnovations are
socially mediated.

\begin{figure*}[h!]
\begin{center}
\includegraphics[width=17.8cm]{figures/figure_3.pdf}
\caption{Cultural traits in oranzees, varying the probability of socially-mediated innovations. Red colour indicates simulation runs that produced more than 38 cultural behaviours; blue colour indicates simulation runs that produces less than 38 cultural behaviours. $S$, $\alpha_e$ and $\alpha_g$ as indicated in the plot. $N=10$ runs for each parameters combination.}
\label{Figure3}
\end{center}
\end{figure*}

Our results show that our model not only accurately reproduces the
number of cultural behaviours (pattern \emph{D}), but also the number of
behaviours classified in the other three patterns (\emph{A}, \emph{B},
\emph{C}) in (1). Figure \ref{Figure4} shows the four patterns produced
in one of the conditions for which we have a good match for cultural
behaviours (\(\alpha_e=0.8;\alpha_g=0.2, S=1\)).

\begin{figure*}[h!]
\begin{center}
\includegraphics[width=11.4cm]{figures/figure_4.pdf}
\caption{Number of behaviours for each of the four patterns (*A*, *B*, *C*, *D*) for the parameters $\alpha_e=0.8;\alpha_g=0.2,S=1$. The red values are the values described for real chimpanzees populations. $N=20$ runs.}
\label{Figure4}
\end{center}
\end{figure*}

Finally, we ran 100 simulations for one of the conditions where we have
a good match for cultural behaviours with (1)
(\(\alpha_e=0.8;\alpha_g=0.2, S=1\)). In each simulation, we recorded,
for each population, the number of behaviours (habitual + customary +
present) that are also classified as cultural (see Figure S4). We find a
small, but significant, correlation between population size and number
of cultural traits (\(p<0.00001,\rho=0.2,N=600\)). In other words, our
model reproduces the effect of cultural accumulation relative to
population size possibly found in real populations - see (9, 13, 14).

\section*{Discussion}\label{discussion}
\addcontentsline{toc}{section}{Discussion}

We developed an individual-based model to examine under which conditions
a distribution of behavioral traits analogous to the distribution
reported in (1) in chimpanzees could emerge, crucially, without allowing
for the existence of any copying mechanism. We implemented several
details of the original study, including realistic demographic and
spatial features, as well as effects of genetic propensity and
ecological availability on the behaviours. Given the widespread
availability of non-copying variants of social learning, we also We also
included socially-mediated reinnovation, where social learning merely
catalyses individual reinnovation (12).

Our main result is that we can reproduce the pattern observed in
populations wild chimpanzees under realistic values of the parameters of
genetic propensity and ecological availability, namely null to medium
importance of genetic variation, and medium to high importance of
ecological variation. Our model cannot precisely determine which exact
values of parameters reproduce real populations of chimpanzees (or other
apes). However, we are confident that the range of values explored, and
the relative ease by which patterns of cultural behaviours similar to
(1) can be produced, strongly suggest that copying is not required for
those patterns to emerge. Therefore, ape-like cultural patterns do not
pinpoint copying abilities. In addition, and as further support to our
results, our model not only reproduces the cultural behavioural
patterns, but also the proportions among the other patterns, i.e.~absent
behaviours, behaviours not achieving habitual frequencies at any site,
and behaviours absent because of ecological factors.

In our model, we focused on the mechanism of socially mediated
reinnovation, that is, we assumed that members of our hypothetical
species, oranzees, had a probability to reinnovate a specific behaviour
stochastically linked to how many other oranzees in the population were
already showing this behaviour. While this is a realistic assumption
(15) and it reproduces in our model the chimpanzees cultural pattern
observed in realistic conditions, our results demonstrate that it is not
necessary. Given certain combinations of parameters, such as higher
genetic and ecological diversities, the same population level pattern
can even be obtained when reinnovation is not socially mediated, i.e if
oranzees are not influenced by the behaviours of the other individuals.

Finally, our model reproduces a correlation between population size and
number of cultural traits in the six populations. The magnitude of the
effect is small, which is to be expected, given that the presence of
this correlation in real populations of (human and non-human) apes is
currently debated ((16)). Again, this correlation is brought about
without copying, so that there is no need to invoke specific
``cultural'' reasons (e.g. (17)) to explain such pattern.

More generally, the results of our models suggest caution when deriving
individual-level mechanisms from population-level patterns (see also
(18, 19)). Cultural systems, as many others, often exhibit equifinality:
the same global state can be produced by different local processes.
Models and experiments are crucial to test the plausibility of
inferences going from global to local properties.

In conclusion, our model strongly suggests that the data available on
the behavioural distributions of chimpanzees populations cannot
demonstrate that chimpanzees possess cultures influenced by copying, let
alone \emph{requiring} copying. This, in turn, may provide an
explanation to why ape cultures are not cumulative.

\showmatmethods
\showacknow
\pnasbreak

\hypertarget{refs}{}
\hypertarget{ref-whiten_cultures_1999}{}
1. Whiten A, et al. (1999) Cultures in chimpanzees. \emph{Nature}
399(6737):682--685.

\hypertarget{ref-henrich_secret_2015}{}
2. Henrich J (2015) \emph{The Secret of Our Success: How Culture Is
Driving Human Evolution, Domesticating Our Species, and Making Us
Smarter} (Princeton University Press, Princeton \& Oxford).

\hypertarget{ref-boyd_different_2017}{}
3. Boyd R (2017) \emph{A Different Kind of Animal: How Culture
Transformed Our Species} (Princeton University Press, Princeton).

\hypertarget{ref-whiten_primate_2000}{}
4. Whiten A (2000) Primate culture and social learning. \emph{Cognitive
Science} 24(3):477--508.

\hypertarget{ref-van_schaik_orangutan_2003}{}
5. Schaik CP van, et al. (2003) Orangutan Cultures and the Evolution of
Material Culture. \emph{Science} 299(5603):102--105.

\hypertarget{ref-mesoudi_what_2018}{}
6. Mesoudi A, Thornton A (2018) What is cumulative cultural evolution?
\emph{Proceedings of the Royal Society B: Biological Sciences}
285(1880):20180712.

\hypertarget{ref-acerbi_cultural_2019}{}
7. Acerbi A (2019) \emph{Cultural Evolution in the Digital Age} (Oxford
University Press, Oxford, New York).

\hypertarget{ref-morin_how_2015}{}
8. Morin O (2015) \emph{How Traditions Live and Die} (Oxford University
Press, London \& New York).

\hypertarget{ref-lind_number_2010}{}
9. Lind J, Lindenfors P (2010) The Number of Cultural Traits Is
Correlated with Female Group Size but Not with Male Group Size in
Chimpanzee Communities. \emph{PLoS ONE} 5(3).
doi:\href{https://doi.org/10.1371/journal.pone.0009241}{10.1371/journal.pone.0009241}.

\hypertarget{ref-wrangham_why_2000}{}
10. Wrangham RW (2000) Why are male chimpanzees more gregarious than
mothers? A scramble competition hypothesis. \emph{Primate Males: Causes
and Consequences of Variation in Group Composition} (Cambridge
University Press, Cambridge), pp 248--258.

\hypertarget{ref-hill_mortality_2001}{}
11. Hill K, et al. (2001) Mortality rates among wild chimpanzees.
\emph{Journal of Human Evolution} 40(5):437--450.

\hypertarget{ref-bandini_spontaneous_2017}{}
12. Bandini E, Tennie C (2017) Spontaneous reoccurrence of ``scooping'',
a wild tool-use behaviour, in naïve chimpanzees. \emph{PeerJ} 5:e3814.

\hypertarget{ref-whiten_evolution_2007}{}
13. Whiten A, Schaik CP van (2007) The evolution of animal ``cultures''
and social intelligence. \emph{Philosophical Transactions of the Royal
Society B: Biological Sciences} 362(1480):603--620.

\hypertarget{ref-kuhl_human_2019}{}
14. Kühl HS, et al. (2019) Human impact erodes chimpanzee behavioral
diversity. \emph{Science} 363(6434):1453--1455.

\hypertarget{ref-tennie_evidence_2010}{}
15. Tennie C, Call J, Tomasello M (2010) Evidence for Emulation in
Chimpanzees in Social Settings Using the Floating Peanut Task.
\emph{PLoS ONE} 5(5).
doi:\href{https://doi.org/10.1371/journal.pone.0010544}{10.1371/journal.pone.0010544}.

\hypertarget{ref-vaesen_population_2016}{}
16. Vaesen K, Collard M, Cosgrove R, Roebroeks W (2016) Population size
does not explain past changes in cultural complexity. \emph{Proceedings
of the National Academy of Sciences of the United States of America}
113(16):E2241--2247.

\hypertarget{ref-henrich_demography_2004}{}
17. Henrich J (2004) Demography and Cultural Evolution: How Adaptive
Cultural Processes can Produce Maladaptive Losses: The Tasmanian Case.
\emph{American Antiquity} 69(2):197--214.

\hypertarget{ref-acerbi_conformity_2016}{}
18. Acerbi A, Van Leeuwen EJ, Haun DB, Tennie C (2016) Conformity cannot
be identified based on population-level signatures. \emph{Scientific
reports} 6:36068.

\hypertarget{ref-barrett_equifinality_2019}{}
19. Barrett BJ (2019) Equifinality in empirical studies of cultural
transmission. \emph{Behavioural Processes} 161:129--138.



% Bibliography
% \bibliography{pnas-sample}

\end{document}

