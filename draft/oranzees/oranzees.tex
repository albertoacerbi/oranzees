\documentclass[9pt,twocolumn,twoside,]{pnas-new}

%% Some pieces required from the pandoc template
\providecommand{\tightlist}{%
  \setlength{\itemsep}{0pt}\setlength{\parskip}{0pt}}

% Use the lineno option to display guide line numbers if required.
% Note that the use of elements such as single-column equations
% may affect the guide line number alignment.


\usepackage[T1]{fontenc}
\usepackage[utf8]{inputenc}


\templatetype{pnasresearcharticle}  % Choose template

\title{Culture in oranzees}

\author[a,1]{Alberto Acerbi}
\author[b]{William Snyder}
\author[b]{Claudio Tennie}

  \affil[a]{Centre for Culture and Evolution, Department of Life Sciences, Brunel
University London, Uxbridge, UB8 3PH, United Kingdom}
  \affil[b]{Faculty of Science, Department for early prehistory and quaternary
ecology, University of Tübingen, Schloß Hohentuebingen, Burgsteige 11,
72070, Tübingen, Germany}


% Please give the surname of the lead author for the running footer
\leadauthor{Acerbi}

% Please add here a significance statement to explain the relevance of your work
\significancestatement{Authors must submit a 120-word maximum statement about the significance
of their research paper written at a level understandable to an
undergraduate educated scientist outside their field of speciality. The
primary goal of the Significance Statement is to explain the relevance
of the work in broad context to a broad readership. The Significance
Statement appears in the paper itself and is required for all research
papers.}


\authorcontributions{Please provide details of author contributions here.}

\authordeclaration{Please declare any conflict of interest here.}


\correspondingauthor{\textsuperscript{} }

% Keywords are not mandatory, but authors are strongly encouraged to provide them. If provided, please include two to five keywords, separated by the pipe symbol, e.g:
 \keywords{  one |  two |  optional |  optional |  optional  } 

\begin{abstract}
Please provide an abstract of no more than 250 words in a single
paragraph. Abstracts should explain to the general reader the major
contributions of the article. References in the abstract must be cited
in full within the abstract itself and cited in the text.
\end{abstract}

\dates{This manuscript was compiled on \today}
\doi{\url{www.pnas.org/cgi/doi/10.1073/pnas.XXXXXXXXXX}}

\begin{document}

% Optional adjustment to line up main text (after abstract) of first page with line numbers, when using both lineno and twocolumn options.
% You should only change this length when you've finalised the article contents.
\verticaladjustment{-2pt}

\maketitle
\thispagestyle{firststyle}
\ifthenelse{\boolean{shortarticle}}{\ifthenelse{\boolean{singlecolumn}}{\abscontentformatted}{\abscontent}}{}

% If your first paragraph (i.e. with the \dropcap) contains a list environment (quote, quotation, theorem, definition, enumerate, itemize...), the line after the list may have some extra indentation. If this is the case, add \parshape=0 to the end of the list environment.

\acknow{Please include your acknowledgments here, set in a single paragraph.
Please do not include any acknowledgments in the Supporting Information,
or anywhere else in the manuscript.}

Culture, the transmission of knowledge, technologies, and beliefs from
individual to individual, and from one generation to the other, is key
to explain the extraordinary ecological success of our species (1).
Which cognitive abilities underpin humans cultural capacities, and how
these abilities affect the evolution of culture itself are among the
most pressing question of evolutionary human science.

Many other species, besides humans, are able to transmit information
socially, from primates (3) to fish (4) and even insects (5).
Differently from these species, however, humans create cumulative
culture. While there are various definitions of cumulative culture (6),
some of its characteristics are broadly accepted. Cumulative culture
requires the accumulation of cultural traits (more cultural traits are
present at time \emph{t} than at time \emph{t-1}), their improvement
(cultural traits are time \emph{t} are more effective than traits at
time \emph{t-1}), and ratcheting (the innovation of cultural traits at
time \emph{t} depends on the presence of other traits at time
\emph{t-1}).

Not all human cultural traditions need to be supported bu faithful
copying (7), but a species capable of cumulative culture needs to be
able to transmit and preserve accurately \emph{new} information.
Experiments have indeed shown that humans are capable of that, and they
routinely do it. More controversial is the claim that other species do,
especially outside of experimental scenarios built on purpose. Claims
regarding the existence of animal cultures based on faithful copying
raise a puzzling question: if other species can and do copy, while they
did not develop comparable cumulative cultures? There are only two
possible answer to this question. Either they do not copy, or faithful
social learning does not automatically lead to cumulative culture.

Besides experiments, satisfying answers to this question should come
form observations of real-life instances of culture among non-human
species. Primatologists have claimed the existence of copying-based
culture from observations of wild populations. (8) examined the
population-level distribution of behavioral traits in populations of
chimpanzees at six sites, and it argued that this distribution proved
the existence of different cultures in these populations. While we can
not replicate wild observations changing their parameters, or re-run
history to see the likelihood of a particular outcome, modelling can
help to assess the robustness of such claims.

We developed an individual-based model to assess whether the patterns
observed in the wild and described in (8) actually warrant the existence
of culture supported by faithful copying abilities. We reproduced, in
our model, several details of the original study, including realistic
demographic and spatial features, and effects of ecological availability
and genetic predisposition, to investigate whether an equivalent
distribution of behavioral traits could emerge in the simulated
populations. While our simulated species, \emph{oranzees}, can be
influenced by social cues present in the environment (as widespread in
the animal kingdom), we did not model any high-fidelity social learning
mechanisms.

Our results show that, under realistic values of the main parameters, we
can reproduce the distribution of behavioral traits found in (8). In
other words, as oranzees can and do show cultural patterns resembling
wild ape patterns, this is proof that such patterns cannot be counted as
evidence that copying takes place. This provides an answer to the
puzzling question we presented above: copying is necessary and leads to
cumulative culture, but ape cultures are not based on it.

\section*{Materials and methods}\label{materials-and-methods}
\addcontentsline{toc}{section}{Materials and methods}

We built an individual-based model that reproduces a world inhabited by
six populations of ``oranzees'', an hypothetical ape species. The model
is space-explicit: the oranzees populations are located at relative
positions analogous to the six chimpanzees sites in (8). This is
important to determine the genetic predispositions and ecological
availabilities associated to their behaviours (see below). Population
sizes are also taken from the sites in (8). Following (9), we use data
from (10), and we define population sizes as
\(N=\{20;42;49;76;50;95\}\).

Oranzees are subject to an age-dependent birth/death process, broadly
inspired by descriptions in (11). A time step \(t\) of the simulation
represents a month in oranzees' life. From when they are 25 years old
(\(t=300\)), there is a 1\% probability an oranzee will die each month,
or they die when they are 60 years old (\(t=720\)). The number of
individuals in the population is fixed, so each time an oranzee dies is
replaced by a newborn.

A newborn oranzee does not have any behaviour. Behaviours can be
innovated at each time step. The process of innovation is influenced by:
(i) the oranzees `state', which depends from the behaviours an
individual already possesses, (ii) the frequency of the behaviours
already present in the population (``socially-mediated innovation''),
and (iii) the genetic propensity and ecological availability associated
to the behaviour. At the beginning of the simulations, the populations
are randomly initialised with individuals between 0 and 25 years old.

\subsection*{State}\label{format}
\addcontentsline{toc}{subsection}{State}

In the oranzees world, 64 behaviours are possible. Behaviours are
divided in two categories, namely 32 social and 32 food-related
behaviours.

In the case of social behaviours, we assume four sub-categories, each
with eight possible different behaviours, that serve the same goal.
Oranzees' state is based on how many of the four goals are fulfilled. A
goal is considered fulfilled if an oranzee has at least one behaviour
out of the eight in the sub-category. An oranzee has a state value of
\(0.25\) if, for example, has at least one behaviour among the first
eight behaviour, and none of the others, and a state value of \(1\) if
there is at least one behaviour in each sub-category.
\(p_\text{social}\), the probability to innovate a social behaviour, is
drawn from a normal distribution with mean equal to
\(1-state_\text{social}\).

Food-related behaviours are analogously divided in sub-categories, with
the differences that there is a variable number of behaviours in each
sub-category, and that sub-categories are associated to two different
`nutrients', \emph{Y} and \emph{Z}. The idea is that individuals need to
balance their nutritional intake, so that their optimal diet consist in
a roughly equal number of foodstuff for one and the other nutrient. The
state, for food-related behaviours, depends on the total amount of food
\emph{and} on the balance between nutrients, and it is calculated as the
sum of each sub-category fulfilled (as above, for this happening there
needs to be at least one behaviour) minus the difference between the
number of sub-categories providing nutrient \emph{Y} and the number of
sub-categories providing nutrient \emph{Z}. We normalize the state
between \(0\) and \(1\), and, as above \(p_\text{food}\) is then
calculated as \(1-state_\text{food}\). (Further details in SI).

\subsection*{Socially-mediated innovation}\label{format}
\addcontentsline{toc}{subsection}{Socially-mediated innovation}

At each time step, all oranzees have a probability of innovation for
social and food-related behaviours calculated as described above. The
specific behaviour an oranzee will innovate may depend both on the
frequency of the behaviours already present in the population, and on
the ecological availability and genetic propensity associated to the
behaviour. A further parameter of the model, \(S\), controls the
probability that each innovation is socially-mediated. When an
innovation is socially-mediated, the probability of innovating each
behaviour \(B_i\) is weighted by its proportional instances in the
population, among the behaviours of the same category, so that common
behaviours are more likely to be innovated.

When the innovation is not socially-mediated, the probability of
innovating each behaviour is random. Only one behaviour per category can
be innovated at each time step.

\subsection*{Genetic propensity and ecological
availability}\label{format}
\addcontentsline{toc}{subsection}{Genetic propensity and ecological
availability}

The behaviour selected in the previous step is actually innovated
according to its genetic propensity and, in case of food-related
beahviours, ecological availability.

Genetic propensity is a probability \(p_g(0,1)\), assigned independently
to each of the 64 behaviours. A parameter of the model, \(\alpha_g\),
determines the probability that the genetic propensity of each behaviour
is equal for all the six populations or is different.

If is equal, \(p_g\) is randomly drawn. If different, we assign it using
a geographical gradient. We choose a random point and calculate its
distance to each population. Distances are then transformed to \(p_g\)
by rescaling them between 0 and 1, so that for the farther site
\(p_g=0\) i.e.~the associated behaviour will be impossible to express
(see SI). Notice that \(\alpha_g=0\) does not mean that there are not
genetic influences on the behaviours, but that there are no
\emph{differences} between the populations with this respect.

Ecological availability is a probability \(p_e(0,1)\) that represents
the likelihood of finding a resource, or its nutritional value, in each
site. Ecological availability is assigned only to food-related
behaviours, and it is calculated in the same way of \(p_g\), using the
parameter \(\alpha_e\) to determine the probability of ecological
availability being different in the six populations.

\subsection*{Model's output}\label{format}
\addcontentsline{toc}{subsection}{Model's output}

We run simulation for \(t_\text{max}=6000\) (corresponding to 500 years
of oranzee-time). For each simulation, following (8), we classify each
behaviour, in each population, as:

\begin{itemize}
\item
  \emph{customary}: a behaviour observed in over 50\% of individuals in
  at least one age class (see SI for how age classes are defined in our
  model).
\item
  \emph{habitual}: a behaviour observed in at least two individuals over
  all the population.
\item
  \emph{present}: a behaviour observed in at least one individual over
  all the population.
\item
  \emph{absent}: a behaviour never observed.
\item
  \emph{ecological explanations} is a behaviour that is absent because
  of local ecological features (i.e., in our model, associated to
  \(p_e=0\)).
\end{itemize}

Notice the last category in (8) (\emph{unknown}, i.e. ``the behaviour
has not been recorded, but this may be due to inadequacy of relevant
observational opportunities'') does not apply in our case.

Finally, we calculate the same ``patterns'' described in (8):

\begin{itemize}
\item
  \emph{A}: patterns absent at no site.
\item
  \emph{B}: patterns not achieving habitual frequencies at any site.
\item
  \emph{C}: patterns for which any absence can be explained by local
  ecological factors.
\item
  \emph{D}: patterns customary or habitual at some sites yet absent at
  others, with no ecological explanation, i.e.~the behaviours defined as
  ``cultural''.
\end{itemize}

\section*{Results}\label{results}
\addcontentsline{toc}{section}{Results}

We are particularly interested in the realistic parameter conditions of
moderate to high environmental variability (\(\alpha_e=(0.5,1)\)) and
zero to moderate genetic differences (\(\alpha_g=(0,0.5)\)). We run 20
simulations for each combination (for a total of 600 runs). For all,
innovation is socially-mediated (\(S=1\)). The results show that various
combinations of parameters produces a number of cultural behaviours
(pattern \emph{D}) consistent with the 38 found in (8), in absence of
any explicit copying mechanism implemented (see Figure \ref{Figure1}).

\begin{figure*}[h!]
\begin{center}
\includegraphics[width=17.8cm]{figures/figure_1.pdf}
\caption{Cultural traits in oranzees, varying ecological and genetic diversity. Red colour indicates simulation runs that produced more than 38 cultural behaviours; blue colour indicates simulation runs that produces less than 38 cultural behaviours. For all simulations, $S=1$, $\alpha_e$ and $alpha_g=0$ as indicated in the plot. $N=20$ runs for each parameters combination.}
\label{Figure1}
\end{center}
\end{figure*}

We also analyse the effect of the parameter \(S\) (proportion of
socially-mediated innovations), in three conditions (see Figure
\ref{Figure2}): (a) no genetic differences and intermediate ecological
differences (compare to the high-left angle of Figure \ref{Figure1},
where with \(S=1\) simulations produce less than 38 cultural
behaviours), (b) good match with (8), and (c) intermediate genetic
differences and high ecological differences (compare to the low-right
angle of Figure \ref{Figure1}, where with \(S=1\) simulations produce
more than 38 cultural behaviours). As expected, decreasing \emph{S},
decreases the number of cultural behaviours. Conditions where, with
\(S=1\), there were more than 38 cultural behaviours could still produce
results analogous to (8), if not all innovations are socially mediated.

\begin{figure*}[h!]
\begin{center}
\includegraphics[width=17.8cm]{figures/figure_2.pdf}
\caption{Cultural traits in oranzees, varying the probability of socially-mediated innovations. Red colour indicates simulation runs that produced more than 38 cultural behaviours; blue colour indicates simulation runs that produces less than 38 cultural behaviours. $S$, $\alpha_e$ and $alpha_g=0$ as indicated in the plot. $N=10$ runs for each parameters combination.}
\label{Figure2}
\end{center}
\end{figure*}

Our results show that our model not only reproduces the number of
cultural behaviours (pattern \emph{D}), but also the number of
behaviours classified in the other three patterns (\emph{A}, \emph{B},
\emph{C}) in (8). Figure \ref{Figure3} show the four patterns produced
in one of the conditions for which we have a good match for cultural
behaviours (\(\alpha_e=0.8;\alpha_g=0.2, S=1\)).

\begin{figure*}[h!]
\begin{center}
\includegraphics[width=11.4cm]{figures/figure_3.pdf}
\caption{Number of behaviours for each of the four patterns (*A*, *B*, *C*, *D*) for the parameters $\alpha_e=0.8;\alpha_g=0.2,S=1$. The red values are the values described for real chimpanzees populations. $N=20$ runs.}
\label{Figure3}
\end{center}
\end{figure*}

Finally, we run 100 simulations for the same condition where we have a
good match for cultural behaviours with (8)
(\(\alpha_e=0.8;\alpha_g=0.2, S=1\)). In each simulation, we recorded,
for each population, the number of behaviours (habitual + customary +
present) that are also classified as cultural (see Figure
\ref{Figure4}). We find a small but significant correlation between
population size and number of cultural traits
(\(p<0.00001,\rho=0.2,N=600\)).

\begin{figure*}[h!]
\begin{center}
\includegraphics[width=11.4cm]{figures/figure_4.pdf}
\caption{Number of cultural behaviours for each population for the parameters $\alpha_e=0.8;\alpha_g=0.2,S=1$. The blue line is a linear fit of the data. $N=100$ runs.}
\label{Figure4}
\end{center}
\end{figure*}

\section*{Discussion}\label{discussion}
\addcontentsline{toc}{section}{Discussion}

TO DO

\showmatmethods
\showacknow
\pnasbreak

\hypertarget{refs}{}
\hypertarget{ref-henrich_secret_2015}{}
1. Henrich J (2015) \emph{The Secret of Our Success: How Culture Is
Driving Human Evolution, Domesticating Our Species, and Making Us
Smarter} (Princeton University Press, Princeton \& Oxford).

\hypertarget{ref-boyd_different_2017}{}
2. Boyd R (2017) \emph{A Different Kind of Animal: How Culture
Transformed Our Species} (Princeton University Press, Princeton).

\hypertarget{ref-whiten_primate_2000}{}
3. Whiten A (2000) Primate culture and social learning. \emph{Cognitive
Science} 24(3):477--508.

\hypertarget{ref-brown_social_2003}{}
4. Brown C, Laland KN (2003) Social learning in fishes: A review.
\emph{Fish and Fisheries} 4(3):280--288.

\hypertarget{ref-leadbeater_social_2007}{}
5. Leadbeater E, Chittka L (2007) Social Learning in Insects --- From
Miniature Brains to Consensus Building. \emph{Current Biology}
17(16):R703--R713.

\hypertarget{ref-mesoudi_what_2018}{}
6. Mesoudi A, Thornton A (2018) What is cumulative cultural evolution?
\emph{Proceedings of the Royal Society B: Biological Sciences}
285(1880):20180712.

\hypertarget{ref-morin_how_2015}{}
7. Morin O (2015) \emph{How Traditions Live and Die} (Oxford University
Press, London \& New York).

\hypertarget{ref-whiten_cultures_1999}{}
8. Whiten A, et al. (1999) Cultures in chimpanzees. \emph{Nature}
399(6737):682--685.

\hypertarget{ref-lind_number_2010}{}
9. Lind J, Lindenfors P (2010) The Number of Cultural Traits Is
Correlated with Female Group Size but Not with Male Group Size in
Chimpanzee Communities. \emph{PLoS ONE} 5(3).
doi:\href{https://doi.org/10.1371/journal.pone.0009241}{10.1371/journal.pone.0009241}.

\hypertarget{ref-wrangham_why_2000}{}
10. Wrangham RW (2000) Why are male chimpanzees more gregarious than
mothers? A scramble competition hypothesis. \emph{Primate Males: Causes
and Consequences of Variation in Group Composition} (Cambridge
University Press, Cambridge), pp 248--258.

\hypertarget{ref-hill_mortality_2001}{}
11. Hill K, et al. (2001) Mortality rates among wild chimpanzees.
\emph{Journal of Human Evolution} 40(5):437--450.



% Bibliography
% \bibliography{pnas-sample}

\end{document}

